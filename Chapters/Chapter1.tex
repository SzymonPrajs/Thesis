% Chapter 1

\chapter{Introduction} % Write in your own chapter title
\label{Chapter1}
\lhead{Chapter 1. \emph{Introduction}} % Write in your own chapter title to set the page header


\section{SN classification}
This is where the introduction will go

\subsection{SN\,Ia}
text

\subsection{SN\,II}
text

\subsection{SN\,Ib/c}
text

\section{Superluminous Supernovae}
text

\section{SN Surveys}
text

\subsection{Differential photometry}
to Forced photometry works on the principle of selecting high quality, low Point Spread Function (PSF), reference images which are then downgraded in quality to match that of the science images containing the transients. The reference images must not contain any SN light hence they usually either predate the explosion epoch of the SN use images takes several years after the explosion epoch. Once the science and reference images are matched in image quality they are warped using SWARP [CITE] so that they photometric solutions also match. After this, the reference image is subtracted from the science image. The difference here between the quick reduction photometry and force photometry is that instead of measuring the centroids of the point sources in one image and extracting its photometry the centroids are measured in the entire series of images and then an aperture matching that of the PSF of the image is used to extract the flux count.

\section{Cosmology and Distance measurements}
text

\section{Thesis overview}
