\chapter{Introduction}  \label{Chapter1}
\lhead{Chapter 1. \emph{Introduction}}

Supernoavea (SN) are extremely luminous stellar explosions. At their brightest point, they can outshine the galaxy that gave birth to them making them visible far back in distance and time. SNe have been obsrved by stargazers since the the dawn of our kind. By studying nebulea created as remnants of past explosions as well as ancient scripts we know that human history is filled with observations of 'guest stars' most famously in 1083 where the SN now seen as the Crab nebula has been recorded in China, seen to be brighter than the full moon at night and visible during the day, as well as the Tycho SN and XXXX in the 17th century.

Despite their early observations, the term `Supernova' was not cointed until 1934 when Baade \& Zwicky were able to estimate their absolute magnitudes. Using Cepheid variables, they measured the distance to several SN host galaxies and found them to be significantly greater then those of classical `Novae'. The prefix `super' reflects their absolute luminocity which much be much greater than that of novea to explain their bright observed liminosities.

\section{SN classification}
At that point of their discovery, little has been known about the properties and physical origin of these extreme objects, largely due to the sparity of their observations. In the years to come many generations of astronomers build increasingly larger and more sensitive surveys, observing thousands of SNe to date. With this came our understanding of the various types and flavours of SNe and the breath of variation in the objects that can give
rise to such brilliant explosions.

\subsection{SN\,Ia}
Perhaps the most well understood and most heavily studies class are the thermonuclear SNe. While a number of their subclasses exists (amongst others; SN\,.Ia, SN\'Iax, SN\,Ia-91T, SN\,Ia-91bg) all of this objects likely share a similar physical origin as the main SN\,Ia class. There competing theories explaining their origin both describe an thermonuclear explosion of a white dwarf (WD) star, but differ in the mechanism which triggers such event. In the single-degenerate scenario a WD, in a binary system, accreates matter from a main-sequence or red giant star until it reaches the Chandrasekhar limit at which point the electron degeneracy presure can no longer support the star against a gravitational collapse resulting in an increase in pressure and subsequent thermonuclear ignition of its core. The alternative explanation suggests that instead of a single WD, a double WD system comes in interaction resulting in their `collision' which resulting in a powerful shockwave which triggers a sub-Chandrasekhar mass explosion.

Both of the above mechanisms result in a production of $\sim$0.7Msun of Ni56 which decays radioactively to Co56 and Fe56 producing a vast quantity of high energy Gamma radiation, which is subsequently reproduced by the SN ejecta into visible light. Cosmology


\subsection{SN\,II}
The origin of SN\,II in many ways mirrors that of SN\,Ia; both are the result of the end point of the evolution and the death of their progenitor star. SN\,II are born when a $>$8Msun star exhausts all of its nuclear fuel at the end of the iron burning phase. At that point the core is supported purely by the electron degeneracy pressure and collapses shortly afterwards into neutron star under the gravitational pressure of the outter layers of the star. The infalling shells rebound of the, now, solid core and are further energised and accelerated by either the neutrinos released in the collapse of the degenerate core or jets formed due to the accreation of the infalling matter onto the core.

In this process, only a small amount of the Ni56 is created which alone could not explain neither the luminosity of the class of SNe nor their light curve morphologies which are offten associated with a sharp rise followed by either a long platue phase (SN\,IIP) or a linear decline (SN\,IIL). These are the effect of Hydrogen recombination (ionised to neutral) in the outter layer of the ejecta, resulting in a blackbody like spectrum with prominant P-cygni profiles visible in the spectra. Additionally, some objects also show narrow emission lines (SN\,IIn), of mostly hydrogen, which are the result of the iteractions between the SN ejecta and extended material ejected by the progenitor star some time before the main event.

\subsection{SN\,Ib/c}
SN\,Ib/c have a very similar origins to SN\,II. They are also a result of a core collapse of a giant star but significantly they originate from larger, yet more stripped stars. These objects are often referred to as Stripped-Envelope supernovae reflicting the fact that no hydrogen (SN\,Ib) nor hellium (SN\,Ic) are visible in their spectra as they have been stripped from the surface of the star a relatively long time period of time before the main SN event. Some theories suggest that the stipping cannot be explained solely using wind and corronal ejections and must be a result of an interaction with a companion star.

Thanks to the increased mass of the progenitor star, these objects often result in a increased amount of Ni56 production resulting in a higher luminocity and a morphology which in the extreme cases can closely reseble that of a SN\,Ia. Spectroscopically, this class of SNe shows a strong formation of Oxygen and Carbon and well as small quantities of other, intermediate mass elements.

\subsection{Other subclasses}
Outside of the main, most commonly observed, subclasses of SNe lives a number of rares and more exodic transients. Amonst these, there is a number of intermediate classes of CCSNe that originate at the boundries of the progenitor scenarios that lead to the main classes. This includes SNIIb, SNIbn and SNIbc amongst others, however, an overlap also exists between the SN\,Ia and the interacting CCSN. SN\,Ia-CSM are one of the most luminous classes of SNe as the extreme brightness of SN\,Ia is enhanced further by the interaction of the ejecta with a layer of circumstellar material (CSM) likely ejected by the companion star.

As SN surveys become more sensitive and sophisticated,  the number of SN classes known to use increases too. In recent years, higher cadance SN searches have lead to the discovery of a new class of Rapidly Evolving Transients (RAT) with an extreme variation in their peak luminosity ranging from -15 < M < -22. Little is yet known about their physical origins, however, their observations suggest a featureless blackbody-like spectrum, often associated with high, in the early phases, temperatures and rapid cooling. A possible interpretation of these objects is an object which undergoes a direct collapse to a black hole, not producing any Ni56 which explains the lack of a slowly declining light curve. The observed light curve would be then a result of the interaction between the SN shock with an extended shall of dense wind \citep{Piro2015}, similarly to an effect sometimes observed in superluminous supernovae (SLSN).

\section{Superluminous Supernovae}
Superluminous supernovae (SLSNe) are a recently identified class of transients defined as events with an absolute magnitude brighter than $-$21 ($M<-21$) \citep{Gal-Yam2012}. They appear 10-100 times brighter than normal supernova events, and form at least two distinct classes: SLSNe-II, which show signatures of interaction with CSM via hydrogen and other lines \citep{2007ApJ...659L..13O,2007ApJ...666.1116S,2011ApJ...735..106D}, and SLSNe-I (or SLSNe-Ic), which are hydrogen-poor \citep{Quimby2011}. While SLSNe-II may naturally be explained as an extension of the fainter SN\,IIn events, the power source behind SLSNe-I remains a subject of debate.

SLSNe-I (SLSN; hence forth) are the protagonists of this thesis. In this section I will introduce the observations that have led to their discovery, their spectroscopic and photometric properties as well as their host galaxies. Furthermore, I will introduce their most commonly accepted formation theories and their effect on the rate of SLSN.

\subsection{Discovery}
The observational properties of SLSNe had a strong impact on the timing of their discovery. As a rare but luminous class of transients, the probability of their detection was low in the early SN surveys in the local universe due to their lower sensitivity and search volume. The first signs of the existance of a new class of extremely luminous SNe came with the discovery of SN2005ap \citep{Quimby2007}, SN2006gy \citep{Ofek2007} and SCP06F6 \citep{Barbary2009}. In each case the distance measurements for the objects have placed them at a luminosity $\sim$100 times brighter than ordinary SNe. However, the low quality of their early light curves and spectra along with a lack other example resulted in these objects being treated more as extremes of the known classes of objects instead of a separate new class.

This picture has evolved dramatically in the last decade with the onset of a number of deep, wide field surveys such as the Texas Supernova Search \citep[TSS;][]{2006PhDT........13Q}, the Palomar Transient Factory \citep[PTF;][]{2009PASP..121.1395L, 2009PASP..121.1334R}, the Supernova Legacy Survey \citep[SNLS;][]{2006A&amp;A...447...31A,2010A&amp;A...523A...7G,2010AJ....140..518P}, the Panoramic Survey Telescope \& Rapid Response System \citep[Pan-STARRS;][]{2010SPIE.7733E..0EK} and the Dark Energy Survey \citep[DES;][]{2005IJMPA..20.3121F}. With an increased sensitivity, longer observing season and a lack of host galaxy selection bias; each one of these surveys was responsible for detecting several SLSNe, jointly shaping our current understanding of this new and exciting area of SNe research.

\subsubsection{Luminous Supernoavae}
The terms `Luminous Supernoavae' and `Superluminous Supernovae' became popular in the literature upon the discovery of SN2007bi \citep{Gal-Yam2009} during the science verification phase of PTF. Its light curve, containing only the single \textit{r}-band filter, lacked the rise time information but demonstrated a very slow decline consitant with the radioactive decay of a large mass of Ni56. At that time it was believed that such event could be a result of a Pair Instability SNe (PISN; \sref{sec:Origins}) and were therefore thought to be a new class of transients.

Following shortly from this, a number of objects were discovered with similar properties. \citet{Quimby2009} presented a sample of SNe detected by PTF along with the first, comprehensive sample of their spectra. While mostly consistant of a blue, featureless blackbody continuum, some absorption lines including C\,II, Mg\,II and O\,II were identified in all spectra in this sample, confirming the redshifts and extreme luminosity of these events.

\subsection{Properties}
Beyond confirming their redshifts, the prominant UV spectral lines can also used to measure the expansion velocity of the photospeheres of SLSNe, measured to be of the order of XXXXX km\,s$^{-1}$, indicating that SLSNe are a class of very energetic events.

....

\subsection{Origins} \label{sec:Origins}
The most popular model in the literature to explain SLSNe involves energy input from the spin-down of a newly-formed magnetar following a CCSN \citep{2010ApJ...717..245K,2010ApJ...719L.204W,2013ApJ...770..128I}, although alternative models involving PISN \citep{2007Natur.450..390W,2015ApJ...814..108Y} or interaction with a hydrogen-free CSM \citep{2011ApJ...729L...6C,2013ApJ...773...76C,2015arXiv151000834S} have also been proposed.

\subsubsection{Magnetar model}
The spin-down of the magnetar model suggests the birth of SLSN at the death of a massive star with a extremely high magnetic field and spin.

\subsection{Host Galaxies}
Additional clues are also provided by the environments in which SLSNe occur: predominantly vigorously star-forming and low-metallicity dwarf galaxies \citep[e.g.,][]{2014ApJ...787..138L,2015MNRAS.449..917L,2016arXiv160504925C}. This preference for low-metallicity environments is supported by the modelling of the SLSN-I spectra, which favours a fairly low metal abundance \citep{2016MNRAS.458.3455M}.

Furthermore, if the strong preference for a young, low-metallicity environment reflects a real physical effect, any evolution in the SLSN rate with redshift should also track the cosmic star-formation and metal enrichment history of the Universe, and the underlying evolving populations of galaxies.

\subsection{Rates}
Of particular note is the rarity of SLSN-I events. It took many years for the first events to be identified as such \citep{2007ApJ...668L..99Q,2009ApJ...690.1358B}, and for the class to be recognised \citep{2011Natur.474..487Q}, in part due to their blue and relatively featureless optical spectra. Even after several years of study, only around 25 well-observed SLSNe-I exist \citep[e.g., see compilations in][]{2014ApJ...796...87I,2015MNRAS.449.1215P,2015MNRAS.452.3869N}. Initial estimates placed the rate of SLSNe-Ic at less than one for every 1000 core collapse supernovae \citep{2011Natur.474..487Q}, and more recent studies are broadly consistent with this \citep{2013MNRAS.431..912Q,2015MNRAS.448.1206M}. However, there has been no direct measurement of the SLSN-I rate for a well-controlled optical transient survey. Such a measurement can provide constraints on progenitor models, as there must, at the very least, be a sufficient number of any putative progenitor system to produce the observed SLSN rate.

\subsection{Connection to Long GRBs}
text

\section{SN Surveys}
One of the most instrumental advances that has lead to the discovery and the development of our understanding surrounding SLSNe is the dawn of the new era of automated, untargetted wide field SN surveys. While there is now a ever growing number of such projects, I endevour to avoid the discussion of individual surveys and instead summerise their general features that impact the study of SLSNe. While it is important to discuss the benefits these projects bring, I would like to focus the attantion of this section on the limitations of the past, current, and future surveys and the impact these have on the work performed in this thesis, involving SLSN classification and their search in the high redshift universe.

\subsection{Cadances}
It is perhapse unsurprising that in the field dominated by the study of SN\,Ia and their use as cosmological probes, the large majority of surveys operated in the last two decade optomise their design to maximise the quality of their relatively short light curves. The cadances, i.e the frequency of observations, are chosen at $\sim$5\,days, sufficient to give each SN\,Ia light curve between three to five points during the critial rise phase, necessary for high precision model fitting used in the cosmological studies. A number of surveys inluding SDSS, PTF, Pan-STARRS, SNLS amonst others conform to this norm.

As rare, slowly evolving and particularly brilliant objects, SLSNe do not require the same observing criteria as their less luminous cousins. In fact, a design that would maximise their detectability would focus on lower cedance and shallower fields and instead use the telescope time to search a larger observing area. Ideally, these fields should also be placed in the parts of the sky visible for a long period of time. While surveys like this were attempted in the past (e.g SUDSS; \sref{sec:SUDSS}), they were met with particularly bad observing conditions and therefore never realising their full potential. As a result, all observations of SLSNe are limited to relatively short light curves ($\sim$5\,months), often observing the SNe at a very high signal to noise despite their low numbers.

The observing conditions have huge effect on the detection and more so on the analysis of SLSNe despite being one of the hardest aspect of survey design to account for. In the past, survey designed to follow SN only would often receive a chaotic cadance where bad weather usually results in less frequent observations. In the case of DES, where there are other components of the survey sheduled from the same pool of alocated time, unfavourable observing conditions for the element of DES requirig high imae quality often result in DES SN light curves having a denser than designed cadance, albeit with the additional data points often having much lower Signal-to-Noise (S/N) ratio.

\subsection{Differential photometry}
to Forced photometry works on the principle of selecting high quality, low Point Spread Function (PSF), reference images which are then downgraded in quality to match that of the science images containing the transients. The reference images must not contain any SN light hence they usually either predate the explosion epoch of the SN use images takes several years after the explosion epoch. Once the science and reference images are matched in image quality they are warped using SWARP [CITE] so that they photometric solutions also match. After this, the reference image is subtracted from the science image. The difference here between the quick reduction photometry and force photometry is that instead of measuring the centroids of the point sources in one image and extracting its photometry the centroids are measured in the entire series of images and then an aperture matching that of the PSF of the image is used to extract the flux count.

\section{Cosmology and Distance measurements}
The fields of Cosmology and SN researh live in a fascinating symbiaotic relationship. Without our understanding of the expansion of the universe, and variability to use it to measure distances to using redshift we would not be able to study SNe at high redshifts where estimating distances using different methods becomes difficult. Similarly, SNe provide one of the most powerful tools for studing the cosmologial parameters of the our universe, famously leading to the discovery of the accelerated expansion of the universe \citep{Riess1998,Perlmutter1997} for which Adam Riess, Brian Shmidt and Saul Perlmutter have been awarded the Nobel Prize in Physics in 2011.

\subsection{Basic Cosmological Model}
The question of the origin of the universe, its age and size have been at the forefront of scientific research since the dawn of time. For many millenia, we have seen our place at the center of known universe, putting us in a special position against other celestial objects. However, as our technology advanced we have built increasingly more sensitive telescopes allowing us to look further in the universe slowly discovering that our home planet is only one of many orbiting the Sun, which is one of billions of similar stars in our galaxy that forms a single spec in the vast structures of the ever expanding and evolving web of the Universe.

One of the most important discoveries on this journey towards our current understading of the origins of the universe came in 1917[???] when [CITET GALAXY REDSHIFT] discovered that the wavelengths of spectral line features observed for a large majority of local galaxies appear to be shifted towards the longer, or redder, wavelengths. This redshift became key to one of the greatest scientific discovery in human history when [CITE HUBBLE] measured distances to these galaxies using Cepheid variables and discovered that the distance, $d$, to the galaxies is proportional to their velocity, $v$, (computed using the redshift as a Doppler shift), which he concluder was due to the expansion of the universe.

\begin{equation}
H_0~=~v/d~\approx~72\,\mathrm{Km}\,\mathrm{s}^{-1}\,\mathrm{Mpc}^{-1}
\end{equation}

Hubble's discovery brought upon the birth of the field of cosmology with a number of theories, describing the evolution of the universe, being proposed in the years immediatelly to follow. After decades of debate, searching of evidence and paradigm shifting discoveries we now know that we live in a universe which begun in the Big Band 13.7 Billion years ago. After a short, initial era where the radiation pressure dominated its expansion to the universe has entered a phase of matter domination before, $\sim$5 Billion years ago, the universe bacame dominated by the misterious Dark Energy which drives its accelerated expansion.

... write more

\subsection{Relationship between redshift and distance}
The definition of distance is not straight forward in the picuture of an expanding universe. 



\section{Thesis overview}
