\chapter{Introduction}  \label{Chapter1}
\lhead{Chapter 1. \emph{Introduction}}

Supernoavea (SN) are extremely luminous stellar explosions. At their brightest point, they can outshine the galaxy that gave birth to them making them visible far back in distance and time. SNe have been obsrved by stargazers since the the dawn of our kind. By studying nebulea created as remnants of past explosions as well as ancient scripts we know that human history is filled with observations of 'guest stars' most famously in 1083 where the SN now seen as the Crab nebula has been recorded in China, seen to be brighter than the full moon at night and visible during the day, as well as the Tycho SN and XXXX in the 17th century.

Despite their early observations, the term `Supernova' was not cointed until 1934 when Baade \& Zwicky were able to estimate their absolute magnitudes. Using Cepheid variables, they measured the distance to several SN host galaxies and found them to be significantly greater then those of classical `Novae'. The prefix `super' reflects their absolute luminocity which much be much greater than that of novea to explain their bright observed liminosities.

\section{SN classification}
At that point of their discovery, little has been known about the properties and physical origin of these extreme objects, largely due to the sparity of their observations. In the years to come many generations of astronomers build increasingly larger and more sensitive surveys, observing thousands of SNe to date. With this came our understanding of the various types and flavours of SNe and the breath of variation in the objects that can give
rise to such brilliant explosions.

\subsection{SN\,Ia}
Perhaps the most well understood and most heavily studies class are the thermonuclear SNe. While a number of their subclasses exists (amongst others; SN\,.Ia, SN\'Iax, SN\,Ia-91T, SN\,Ia-91bg) all of this objects likely share a similar physical origin as the main SN\,Ia class. There competing theories explaining their origin both describe an thermonuclear explosion of a white dwarf (WD) star, but differ in the mechanism which triggers such event. In the single-degenerate scenario a WD, in a binary system, accreates matter from a main-sequence or red giant star until it reaches the Chandrasekhar limit at which point the electron degeneracy presure can no longer support the star against a gravitational collapse resulting in an increase in pressure and subsequent thermonuclear ignition of its core. The alternative explanation suggests that instead of a single WD, a double WD system comes in interaction resulting in their `collision' which resulting in a powerful shockwave which triggers a sub-Chandrasekhar mass explosion.

Both of the above mechanisms result in a production of $\sim$0.7M_sun of Ni56 which decays radioactively to Co56 and Fe56 producing a vast quantity of high energy Gamma radiation, which is subsequently reproduced by the SN ejecta into visible light. Cosmology


\subsection{SN\,II}
The origin of SN\,II in many ways mirrors that of SN\,Ia; both are the result of the death of their progenitor star and involves a degenerate core. SN\,II are born when a $>$8M_sun star exhausts all of its energy at the end of the Iron burning phase in the core. At that point the core is supported purely by the electron degeneracy pressure  originate as a result of the collapse of the 

\subsection{SN\,Ib/c}
text

\section{Superluminous Supernovae}
text

\section{SN Surveys}
text

\subsection{Differential photometry}
to Forced photometry works on the principle of selecting high quality, low Point Spread Function (PSF), reference images which are then downgraded in quality to match that of the science images containing the transients. The reference images must not contain any SN light hence they usually either predate the explosion epoch of the SN use images takes several years after the explosion epoch. Once the science and reference images are matched in image quality they are warped using SWARP [CITE] so that they photometric solutions also match. After this, the reference image is subtracted from the science image. The difference here between the quick reduction photometry and force photometry is that instead of measuring the centroids of the point sources in one image and extracting its photometry the centroids are measured in the entire series of images and then an aperture matching that of the PSF of the image is used to extract the flux count.

\section{Cosmology and Distance measurements}
text

\section{Thesis overview}
