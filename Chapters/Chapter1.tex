\chapter{Introduction}  \label{Chapter1}
\lhead{Chapter 1. \emph{Introduction}}

Supernoavea (SN) are extremely luminous stellar explosions. At their brightest point, they can outshine the galaxy that gave birth to them making them visible far back in distance and time. SNe have been obsrved by stargazers since the the dawn of our kind. By studying nebulea created as remnants of past explosions as well as ancient scripts we know that human history is filled with observations of 'guest stars' most famously in 1083 where the SN now seen as the Crab nebula has been recorded in China, seen to be brighter than the full moon at night and visible during the day, as well as the Tycho SN and XXXX in the 17th century.

Despite their early observations, the term `Supernova' was not cointed until 1934 when Baade \& Zwicky were able to estimate their absolute magnitudes. Using Cepheid variables, they measured the distance to several SN host galaxies and found them to be significantly greater then those of classical `Novae'. The prefix `super' reflects their absolute luminocity which much be much greater than that of novea to explain their bright observed liminosities.

\section{SN classification}
At that point of their discovery, little has been known about the properties and physical origin of these extreme objects, largely due to the sparity of their observations. In the years to come many generations of astronomers build increasingly larger and more sensitive surveys, observing thousands of SNe to date. With this came our understanding of the various types and flavours of SNe and the breath of variation in the objects that can give
rise to such brilliant explosions.

\subsection{SN\,Ia}
Perhaps the most well understood and most heavily studies class are the thermonuclear SNe. While a number of their subclasses exists (amongst others; SN\,.Ia, SN\'Iax, SN\,Ia-91T, SN\,Ia-91bg) all of this objects likely share a similar physical origin as the main SN\,Ia class. There competing theories explaining their origin both describe an thermonuclear explosion of a white dwarf (WD) star, but differ in the mechanism which triggers such event. In the single-degenerate scenario a WD, in a binary system, accreates matter from a main-sequence or red giant star until it reaches the Chandrasekhar limit at which point the electron degeneracy presure can no longer support the star against a gravitational collapse resulting in an increase in pressure and subsequent thermonuclear ignition of its core. The alternative explanation suggests that instead of a single WD, a double WD system comes in interaction resulting in their `collision' which resulting in a powerful shockwave which triggers a sub-Chandrasekhar mass explosion.

Both of the above mechanisms result in a production of $\sim$0.7M_sun of Ni56 which decays radioactively to Co56 and Fe56 producing a vast quantity of high energy Gamma radiation, which is subsequently reproduced by the SN ejecta into visible light. Cosmology


\subsection{SN\,II}
The origin of SN\,II in many ways mirrors that of SN\,Ia; both are the result of the end point of the evolution and the death of their progenitor star. SN\,II are born when a $>$8M_sun star exhausts all of its nuclear fuel at the end of the iron burning phase. At that point the core is supported purely by the electron degeneracy pressure and collapses shortly afterwards into neutron star under the gravitational pressure of the outter layers of the star. The infalling shells rebound of the, now, solid core and are further energised and accelerated by either the neutrinos released in the collapse of the degenerate core or jets formed due to the accreation of the infalling matter onto the core.

In this process, only a small amount of the Ni56 is created which alone could not explain neither the luminosity of the class of SNe nor their light curve morphologies which are offten associated with a sharp rise followed by either a long platue phase (SN\,IIP) or a linear decline (SN\,IIL). These are the effect of Hydrogen recombination (ionised to neutral) in the outter layer of the ejecta, resulting in a blackbody like spectrum with prominant P-cygni profiles visible in the spectra. Additionally, some objects also show narrow emission lines (SN\,IIn), of mostly hydrogen, which are the result of the iteractions between the SN ejecta and extended material ejected by the progenitor star some time before the main event.

\subsection{SN\,Ib/c}
SN\,Ib/c have a very similar origins to SN\,II. They are also a result of a core collapse of a giant star but significantly they originate from larger, yet more stripped stars. These objects are often referred to as Stripped-Envelope supernovae reflicting the fact that no hydrogen (SN\,Ib) nor hellium (SN\,Ic) are visible in their spectra as they have been stripped from the surface of the star a relatively long time period of time before the main SN event. Some theories suggest that the stipping cannot be explained solely using wind and corronal ejections and must be a result of an interaction with a companion star.

Thanks to the increased mass of the progenitor star, these objects often result in a increased amount of Ni56 production resulting in a higher luminocity and a morphology which in the extreme cases can closely reseble that of a SN\,Ia. Spectroscopically, this class of SNe shows a strong formation of Oxygen and Carbon and well as small quantities of other, intermediate mass elements.

\subsection{Other subclasses}
Outside of the main, most commonly observed, subclasses of SNe lives a number of rares and more exodic transients. Amonst these, there is a number of intermediate classes of CCSNe that originate at the boundries of the progenitor scenarios that lead to the main classes. This includes SNIIb, SNIbn and SNIbc amongst others, however, an overlap also exists between the SN\,Ia and the interacting CCSN. SN\,Ia-CSM are one of the most luminous classes of SNe as the extreme brightness of SN\,Ia is enhanced further by the interaction of the ejecta with a layer of circumstellar material (CSM) likely ejected by the companion star.

As SN surveys become more sensitive and sophisticated,  the number of SN classes known to use increases too. In recent years, higher cadance SN searches have lead to the discovery of a new class of Rapidly Evolving Transients (RAT) with an extreme variation in their peak luminosity ranging from -15 < M < -22. Little is yet known about their physical origins, however, their observations suggest a featureless blackbody-like spectrum, often associated with high, in the early phases, temperatures and rapid cooling. A possible interpretation of these objects is an object which undergoes a direct collapse to a black hole, not producing any Ni56 which explains the lack of a slowly declining light curve. The observed light curve would be then a result of the interaction between the SN shock with an extended shall of dense wind \citep{Piro2015}, similarly to an effect sometimes observed in Superluminous Supernovae (SLSN).

\section{Superluminous Supernovae}
SLSNe are a relatively recently discovered, luminous class of transients often described as objects with a peak absolute magnitude, M$_\mathrm{V} < -21$, i.e 5-100 times increase in luminosity over most ordinary SNe.  

\section{SN Surveys}
text

\subsection{Differential photometry}
to Forced photometry works on the principle of selecting high quality, low Point Spread Function (PSF), reference images which are then downgraded in quality to match that of the science images containing the transients. The reference images must not contain any SN light hence they usually either predate the explosion epoch of the SN use images takes several years after the explosion epoch. Once the science and reference images are matched in image quality they are warped using SWARP [CITE] so that they photometric solutions also match. After this, the reference image is subtracted from the science image. The difference here between the quick reduction photometry and force photometry is that instead of measuring the centroids of the point sources in one image and extracting its photometry the centroids are measured in the entire series of images and then an aperture matching that of the PSF of the image is used to extract the flux count.

\section{Cosmology and Distance measurements}
text

\section{Thesis overview}
