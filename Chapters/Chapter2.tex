% Chapter 2

\chapter{Data} % Write in your own chapter title
\label{Chapter2}
\lhead{Chapter 2. \emph{Data}} % Write in your own chapter title to set the page header

In this chapter I discuss the data, surveys and the follow-up facilities used in this thesis. I begin by discussing the Supernova Legacy Survey (SNLS) which was used in the initial part of this thesis and was the source of data for my work on the rate of SLSNe at z$\sim$1, discussed in \cref{Chapter3}. Following from this I discuss the Dark Energy Survey (DES) which provided a bigger and higher redshift sample of transients used in my search for high redshift SLSNe as described in \cref{Chapter5}. I then give a brief overview of the follow-up facilities used in the classifying of SLSNe that were discovered in real time using the techniques discussed in \cref{Chapter5}. Finally I describe the process of collecting and unifying the sample of published SLSNe used as a baseline training dataset throughout this work.

\section{Supernova Legacy Survey}
The Supernova Legacy Survey \citep{Boulade2003,Pritchet2004} was run as part of the Canadian-French-Hawaii Telescope Legacy Survey (CFHT-LS) between 2003 and 2008. Over that time it has proven to be one of the most successful SN surveys to date, observing thousands of transients and spectroscopically confirming a large proportion of that, including over 300 SN-Ia \citep{Perrett2010}, more than 50 core-collapse supernovae as well as measuring their respective rates \citep{Perrett2012,Bazin2009}. SNLS has also discovered two SLSNe at z=1.588 and z=1.50 \citep{Howell2013} which, until recently, have been the highest redshift, spectroscopically identified objects of this class. The principle objective of the survey was to perform a measurement of the cosmological constants $\omega_{\Lambda}$ and $\omega_{m}$ \citep{Astier2006}. This was extremely successful, producing their most precise ground based measurement of its time \citep{Sullivan2006}. This was in a large possible thanks to the observing strategy optimized to maximize the number of high redshift SN-Ia candidates and the thorough spectroscopic follow-up program  which gave a much stronger constrains for the Hubble diagram thanks to the high redshift leverage not possible before with low redshift SN-Ia surveys.

\subsection{Survey Overview}
SNLS used the 4m Canadian-French-Hawaiian Telescope (CFHT) situated on Mauna Kea, Hawaii. It was operated by two teams from Canada and France as part of the CFHT Legacy Survey that composed of a small area, deep SN survey and a shallow, wide galaxy survey aimed at studying the large scale structure of the Universe and the cosmological parameters though the galaxy clustering and weak lensing \citep{Pritchet2004,Astier2006}. Four, $\sim$1 deg$^2$ deep fields have been observed to the limiting magnitude of m$\sim$23.5 using the 400 megapixel MegaCam camera \citep{Boulade2003}. Observations took place over five seasons, each lasting approximately 5 months. In total 202 nights have been allocated to the survey \citep{Pritchet2004}.

\subsection{Cadence and Observations}
SNLS carried out the observations in the \textit{griz} photometric bands, similar in bandpass to the filters used in SDSS \fref{fig:SNLSFilters}. Each field was observed over five to seven periods, each lasting approximately 18 days during the lunar dark times, with an average cadence of 3-4 days \citep{Astier2006,Guy2010}. Where possible due to weather, all bands were observed simultaneously on the same night. As a survey aimed at producing the most densely populated, high quality Hubble diagram, this cadence was optimized to maximize the number of detections of SN-Ia at mid to high redshift (0.2 $\leq$ z $\leq$ 0.9) with a sufficient data quality to light curve model fitting \citep{Pritchet2004}. The quality criteria required the light curves to have two detection before and two after the peak of the SN which the cadence used by SNLS was well suited for. In cases where the observing plan could not be fully executed, due to weather condition or otherwise, the \textit{r} and \textit{i} band observations have been prioritized over the \textit{g} and \textit{z} band observations \citep{Guy2010}. This meant that even in the early stages of the survey or suboptimal weather conditions, the light curves quality remained good enough for SN-Ia analysis, however, this had a strong effect in the later stages of the season where many light curves are missing data in the \textit{g} bands which had an effect on the analysis of the much bluer SLSN in the sample \citep{Prajs2016}.  

\begin{figure}
  \centering
  \includegraphics[scale=0.8]{Figures/Chapter2/SNLS_filters.pdf}
	\caption{Bandpass response for filters used in SNLS compared to a 				spectrum of a SLSN; SNLS06D4eu at z=1.588 discovered by the 			  survey [TODO - BIN UP]}
    \label{fig:SNLSFilters}
\end{figure}

\subsection{Data Reduction}
Several different data reduction pipelines have been used in the analysis of the SNLS data. In this thesis we used a combination of data prepared for public and internal data releases, as well as reanalysis of data using PTFPhot, a custom, pre-existing pipeline designed to improve the data reduction quality \citep{Firth2015}.

\subsubsection{Real-time photometry}
During the live operations of the survey the data was pre-processed by applying flats and biases using the \textsc{Elixir} \citep{Magnier2004} data has been independently analyses by the French and Canadian teams using separate quick-reduction pipelines \citep{Astier2006,Bazin2011}. Both teams aimed to produce lists of new, viable SN candidates within hours of their observations before sending them to the spectroscopic follow-up facilities. Both teams were very successful and produced mostly identical lists which lead to high classification accuracy, largely contributing to the overall success of the survey \citep{Pritchet2004}.

\subsubsection{Forced photometry}
The quick reduction pipelines used in the detection of objects were optimized with speed, rather than accuracy, in mind. Any scientific analysis of the data requires a more precise treatment of the photometry. The data for 'real' transient, defined as as having detections on multiple epochs and in multiple bands, is passed through a much improved, custom, forced photometry pipeline; PTFPhot \citep{Firth2015}. The improvements in the photometry, compared to the quick photometry, come from mainly from the treatment of the PSF of the images. The quality of the science images is measured precisely on a number of stars allowing for the reference images, usually with better quality, to be downgraded to match the quality of the science frames by convolving them with the PSF. Furthermore, the photometry is measured at the centroid for the source determined in a stack of all images as opposed to individual frames, giving rise to the term 'forced' photometry as the positions of the objects are forced to be consistent between all images. The full treatment of the PSF largely decreases the uncertainties associated with the flux measurement, specially for faint sources close to, or below, the detection limit. This analysis has been applied to the data releases that provided the sources and light curves for for the work described in \cref{Chapter3} following other SNLS analysis such as the rates of SN-Ia \citep{Perrett2012} and core-collapse SN \citep{Bazin2009} as well as the cosmological measurements \citep{Astier2006,Sullivan2011}.

\subsection{Spectroscopic Follow-Up}
The success of SNLS cannot be attributed only to the the number of transients discovered by the survey but also to the effort behind the spectroscopic follow-up of the candidates. Between the Very Large Telescope (VLT), Keck, Gemini North and South and Magellan more time has been allocated for spectroscopic follow-up than the time allocated to the photometric survey alone \citep{Pritchet2004}. As a result 322 SN-Ia have been spectroscopically confirmed along with 51 CCSN and two SLSNe \citep{Guy2010,Howell2005,Howell2013}. A large number of the SN confirmed by the survey have been classified before or at their maximum thanks to both the follow-up selection criteria, which correctly identified the most promising candidates \citep{Sullivan2006}, and the readiness of the survey to sacrifice the classification rate in order to improve the quality of the data. While a large number of CCSN and AGN have been targeted for follow-up, which perhaps in the eyes of the primary goal of SNLS could be considered as undesired contaminants, this process also lead to the accidental discovery of two high redshift SLSN. These objects would be almost certainly overlooked if they were targeted post peak as their evolution would have strongly disfavored them as SN-Ia candidates. In turn, the rapid follow-up of SNLS06D4eu has produced a spectrum at a rest-frame phase of -34days which remains to this day as one of the earliest spectrum of a SLSN.        

\subsection{Redshift measurements}
Distance measurements are invaluable in any type of SN analysis beyond just their application in cosmology. SNLS has provided three types of redshift measurements, used as distance probes: SN spectroscopic redshift, host galaxy spectroscopic redshift and photometric redshift estimates. The case of the SN spectroscopic redshift is the most accurate and rarely disputable as lines used to identify the redshift can be confirmed in the spectrographic images to be coincidental with the trace of the SN light. In SNLS this was measured for all classified SN. Furthermore it may sometimes be possible to measure the redshift of an object that has an ambiguous classification as the some spectral features may be strong enough to be detectable despite low S\/N of the continuum.

Similarly to the SN spectroscopy, the host spectroscopy can provide a very high accuracy measurement of the redshift using narrow spectral features. However, it is possible that the host of the object may be misidentified, e.g the redshift might be measured for a bright galaxy apparently coincidental with the SN while the true host was a background galaxy or a dwarf foreground galaxy. Since these errors are relatively rare, we use both the target and host spectroscopic redshifts as absolute. SNLS, similarly to other large SN surveys, was not able to target all potential SN candidates during the live operations of the survey with many unclassified objects, later photometrically selected as good SN candidates after they faded below detection limit. \citet{Lidman2012} used the AAOmega multi fiber spectrograph on the Anglo-Australian Telescope (AAT) to target nearly 700 hosts of SN candidates in two of the four SNLS fields, obtaining redshift for 400 of them.

Beyond the spectroscopic measurement, it is also possible to estimate the redshift using the photometry of the galaxies. The SED of a galaxy usually contains a characteristic sharp break at $\sim$4000\AA which can be observed to transition between filters as the a function of redshift. This is a very powerful technique which, when used with a number of photometric filters, can very accurately estimate the redshift of a galaxy above z>0.3. Below that threshold the SED break is contained within one filter largely increasing the uncertainty of the estimate \citep{Connolly1995}. This technique has been used to estimate redshifts for 500,000 galaxies within the CFHTLS which overlapped with the SNLS fields giving a redshift estimate to nearly all candidates with a detected host \citep{Ilbert2006}.

\subsection{SLSN in SNLS}
SNLS was not an ideal survey for detecting SLSN. Do to the primary objective of the project being the study of SN-Ia, its focus on a small area and relatively dense cadence meant that the volume searched was not sufficient to the detect a large number of SLSN, known to be a very rare class of astronomical events \citep{Cooke2012,Prajs2016,Quimby2013}. Despite this, two events have been discovered; SNLS07D2bs at z=1.50 and SNLS06D4eu at z=1.588 with the remaining as the highest redshift SLSN discovered for a full decade until this record was taken by DES.

Furthermore, after the completion of the survey from individual seasons have been co-added together to create "super" stacks reaching a detection magnitude of m=26.5 \citep{Cooke2012}. Two objects have been detected using this technique that had a light curve behavior similar to that of a SLSN. While it was impossible at that point (several years after the live transient) to spectroscopically identify these objects, host galaxy spectroscopy was obtained for both objects determining redshifts of z=2.05 and z=3.9 \citep{Cooke2012}. This also confirmed that the transients had a luminosities comparable to that of SLSN. While these candidate SLSNe show an interesting precedent for what may be hiding beyond the reach of our current, 4m telescope class, optical surveys, they were not used in any projects within this thesis. The lack of certain confirmation, low cadence in the stacked images light curve and the difference in the detection technique used deemed these objects incompatible with other SLSN data sources \citep{Prajs2016}.

\section{Dark Energy Survey}
The Dark Energy Survey (DES) is the largest cosmology survey operated to date. It is composed of two elements; a wide galaxy survey and a SN component. It's main aim is to perform the most precise, to date, measurement of the cosmological parameters.

\subsection{Survey Setup}
DES uses a purpose build XXX mega-pixel DECam camera mounted on the 4m Blanco telescope at the Cerro-Tollolo observatory in Chile. One of the greatest advantages of DECam over its predecessors is its high sensitivity at red wavelengths allowing for high signal to noise observations in \textit{z} and \textit{y} bands. DES was awarded 500 observing nights over the period of 5 years between 2013 and 2018. Each observing seasons starts in late August and ends at the end of January giving on average a 5 months long observing period.

\subsubsection{Wide Survey}
DES wide survey has been designed to observe 5000 deg$^2$ of the southern sky to the depth of m$\sim$25 in the \textit{ugrizy} filters. The aim of this is to create the largest catalog of galaxies and their associated redshifts to date at an unprecedented resolution and depth with the first public data release contained a catalog of 400 million galaxies from the initial 3 years of the survey. These data can be used to study cosmology via three separate experiments; the Baryon Acoustic Oscillations (BYO), Galaxy Clustering and Weak Lensing [CITE ALL].

\subsubsection{Supernova Survey}
The SN survey as part of DES is a dedicated search optimized to study and discovered the highest number of SN-Ia to date. To maximize the number of discovered SN-Ia 10 fields are observed in the \textit{griz} photometric bands. Eight of these fields are 'shallow' and observe to the depth of m=23.5 while two fields have been chosen to be 'deep' and observe to m=25. The number of deep and shallow fields was chosen as an optimal value that allowed for the largest number of SN-Ia to be discovered at the same time searching for high redshift objects at z>1 [CITE].

The cadence of the DES

As the the Weak Lensing experiment of the wide survey requires very good seeing images in order to resolve the shapes of galaxies, it is that component that receives a priority over the best quality night. In case where the observations are

\subsection{Photometric Redshifts}
With the vast quantities of galaxies detected by DES it is not currently possible, even with the largest integrated field spectrograph, to obtain spectroscopic redshifts for even a small fraction of these objects. A photometric approach must be used to achieve the overall cosmological goals of the survey. While most DES observations are carried out using the \textit{griz} filters \textit{u} and \textit{y} are used, albeit at a lower depth, to aid with the galaxy SED template fitting used to estimate the redshifts. This method is not particularly accurate at low redshifts, however it becomes sufficiently precise at z>0.3 [CITE]. This is due to the steep oxygen[??] drop at 4000\AA [??] shifting into the redder bands amongst other galaxy features.


\section{SUDSS}
The Search Using DECam for Superluminous Supernovae (SUDSS) (PI: Sullivan) was designed as a dedicated SLSN survey working in conjunction with DES. Its aim was to extend the DES observing season from 6 months to $\sim$ 9 months by observing the fields at high airmass until the observations become impossible due the sun constrains. While, do to technical issued \sref{sec:SUDSSIssues}, it has never been used as a tool for searching for new SLSN candidates it has greatly enhanced the light curves of some SLSNe discovered by DES.

\subsection{Observations and Data Reduction}
SUDSS observed XX fields between the end of 1st DES season and start of the 4th season (Periods Pxx to Pxx). The observations were carried out using DECam with an identical setup to DES allowing for the same pre-processing steps to be applied to the data. The data was not directly fed into the DES database nor did it use the force photometry pipeline used by DES for technical reasons. Instead the data was reduced using a custome PTFPhot pipeline [CITE ROB]. For the purpose of unbiased comparison, corresponding DES observations of the SUDSS SLSNe were also reduced using the PTFPhot pipeline.

\subsection{Cadance and Exposures}
\label{sec:SUDSSCadance}
SUDSS was designed with the aim of discovering new SLSN candidates at its very heart. It was initially believed [CITE] that it will be capable of identifying $\sim$ 100 SLSN candidates during its operation. As SLSN are a class of slowly evolving supernova, the DES observing cadance was not retained for SUDSS and instead it was chosen to be two weeks. Due to worse that anticipated weather conditions many of these nights have been lost to cloudy nights resuting in a final cadance closer to 1 month. On avarage three additional observations were made in each field outside of the DES observing season.

A futher difference between the DES and SUDSS observing strategies comes in the exposure times used for the observations. As SLSN are much brighter but rarer events than SN-Ia (which are the main focus of the DES SN survey) the exposure times did not have to match that of the DES deep fields to still search for SLSNe at redshifts greater than z>1, however, they did exceed the relatively short exposure times of DES shallow SN field. In practice the limitig magnitude of SUDSS was similar to that of the DES shallow fields once the unfavorable atmospheric conditions and high air mass have been taken into account.

\subsection{Findings}
\label{sec:SUDSSIssues}
As a result of the poor cadance and lower than expected image quality it was not possible to carry out a dedicated search for SLSN in the SUDSS data alone. Instead, the SUDSS data have been repurposed as an auxilary data source for objects already discovered by DES. This has been extremely sucessful and resulted in a great improvemnt to the light curves of some of the SLSNe discoved by DES. In cases of DES15C3hav and DES15X2hm we were able to determine the explosion date and hence contrain the rise time of these SNe using the SUDSS data points.

The most significant observations came in form of DES14X3taz [CITE MAT] where the auxilary SUDSS data provided the observations of the pivotal rise and peak of this unique "bumpy" SLSN. Without the SUDSS observations a large proportion of the analysis carried out in [CITE] would have not been possible.

\section{DES Spectroscopic Follow-up Facilities}

\section{Literature sample of SLSNe}
Throughout this thesis a sample of SLSN, published in years between 2010 and 2016, has been used to define what a SLSN is. We define SLSN as any hydrogen poor SLSN and do not distinguish between slowly and rapidly evolving objects [CITE INSERRA AND GAL-YAM]. All objects come from a variety of surveys which includes xx PTF, 1 iPTF, 2 SNLS, 3 DES and xx PS1 SNe. The broad mix of surveys means that the objects are not of the same quality, are not in the same photometric systems and do not use the same band passes.

\subsection{Quality cuts}
We devised a set of quality cuts to ensure that the objects used in our analysis meet a minimum standard. As most of our analysis, discussed in the forthcoming chapters, depends on fitting black bodies to the light curve we require that the light curve must contain a minimum of three distinct photometric bands, i.e Even thought they would be modeled independently we would count SDSS and PS1 \textit{r}-band filters as one for the purpose of the quality cut. Furthermore we make a cut on the number of epochs observed per band. In order to use models which depend on the rise and decline time of the supernova we require that there are a minimum of two photometric points before the peak and two points after the peak. As the SN evolve rapidly in temperature and hence colour, the bluer bands peak several days before redder bands. We define the peak as the maximum in the band closest to rest frame U-band. \tref{tab:PubishedSLSNe} shows the list of SLSN used throughout this work that have passed our data quality cuts.

\subsection{Converting and converging photometric systems}
It is standard practice in the SN literature to publish the photometry for studied objects in form of a table of magnitudes per band on given observing nights. Magnitude scale is not, however, the most useful when fitting models to the data. As it is logarithmic in nature, the photometric uncertainties, originally measured in the linear count space and often Gaussian in nature become asymmetric and therefore more difficult to deal with in model fitting. It is a lot easier to convert the magnitudes, M, to flux, $\nu$ using a zero point, $zp$ for a given filter:
\begin{equation}
\label{eq:MagToFlux}
\nu = 10^{-0.4 \times M~+~zp}
\end{equation}
similarly we convert the uncertainties in magnitude to flux using \eqref{eq:MagErrorToFlux}.
\begin{equation}
\label{eq:MagErrorToFlux}
\Delta \nu = 0.921034~\times~10^{-0.4 \times M~+~zp}~\times~\Delta M
\end{equation}
It is customary in literature to quote the magnitudes for the SDSS filters (\textit{ugrizy}) in the AB photometric system and the Johnson filters (\textit{UBVRIJHK}) in the Vega system. If not otherwise stated in the papers these are the systems used for the zero points in conversion between magnitude and flux. If not stated otherwise it is also assumed that the photometry has been corrected for Milky Way extinction which is a standard procedure in most modern surveys [CITE PTF, DES, SNLS etc].

\begin{table}
\begin{center}
  \caption{The training sample of SLSNe-I.}
\label{tab:PubishedSLSNe}
\begin{tabular}{|l|r|l|l|}
\hline
  \multicolumn{1}{|c|}{SN Name} &
  \multicolumn{1}{c|}{Redshift} &
  \multicolumn{1}{c|}{Survey} &
  \multicolumn{1}{c|}{Reference} \\
\hline
  PTF12dam & 0.108 & Palomar Transient Factory & \citep{2013Natur.502..346N}\\
  SN2011ke & 0.143 & Catalina Real-Time Transient Survey & \citet{2013ApJ...770..128I}\\
  &&\& Palomar Transient Factor & \\
  SN2010gx & 0.230 & Palomar Transient Factory & \cite{2010ApJ...724L..16P}\\
  SN2013dg & 0.265 & Catalina Real-Time Transient Survey & \cite{2014MNRAS.444.2096N} \\
  PS1-11ap & 0.524 & Pan-STARRS & \cite{2014MNRAS.437..656M}\\
  DES14X3taz & 0.608 & Dark Energy Survey & \cite{2016ApJ...818L...8S} \\
  PS1-10bzj & 0.650 & Pan-STARRS & \cite{2013ApJ...771...97L}\\
  DES13S2cmm & 0.663 & Dark Energy Survey & \cite{2015MNRAS.449.1215P} \\
  iPTF13ajg & 0.741 & intermediate Palomar Transient Factory &\cite{2014ApJ...797...24V}\\
  SNLS-07D2bv & 1.500 & SNLS &\cite{2013ApJ...779...98H}\\
  SNLS-06D4eu & 1.588 & SNLS &\cite{2013ApJ...779...98H}\\
\hline\end{tabular}
\end{center}
\end{table}
