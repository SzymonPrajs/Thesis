% Chapter 2

\chapter{Data} % Write in your own chapter title
\label{Chapter2}
\lhead{Chapter 2. \emph{Data}} % Write in your own chapter title to set the page header

In this chapter I discuss the data, surveys and the follow-up facilities used in this thesis. I begin by discussing the Supernova Legacy Survey (SNLS) which was used in the early part of my thesis and was the source of data for a project investigating the rate of SLSNe at z$\sim$1 as discussed further in Chapter 3[CITE THIS PROPERLY]. Following from this I discuss the Dark Energy Survey (DES) which was used in my search for SLSN at high redshift as described in Chapter 5 [CITE THIS PROPERLY AGAIN]. I then give a brief overview of the follow-up facilities used in the classifying of SLSNe that were discovered in real time using the techniques discussed in Chapter 5 [CITE]. Finally I describe the process of collecting and unifying the sample of published SLSNe used as a baseline dataset throughout this work.

\section{Supernova Legacy Survey}
The Supernova Legacy Survey (SNLS) [CITE] was one of the most successful SN surveys to date. It was run as part of the Canadian-French-Hawaii Telescope Legacy Survey (CFHT-LS) between 2003 and 2007. Over that time it has observed and spectroscopically confirmed over 700 SN-Ia [CITE AND CONFIRM NUMBER], giving one of the best measurement of their rate [CITE PERETT] as well as produced the most accurate, at that point, measurement of the cosmological parameters $\omega_{\Lambda}$ and $\omega_{m}$ [CITE]. This was in large way possible thanks to the observing strategy designed specifically to search for high redshift SN-Ia and the thorough spectroscopic follow-up program which has targeted a large portion of SN-Ia candidates.

\subsection{Survey Overview}
SNLS used the 4m Canadian-French-Hawaiian Telescope (CFHT) situated on Mauna Kea, Hawaii. It was operated by two, mostly independent teams from Canada and France. Four, 1 deg$^2$ deep fields have been observed to the limiting magnitude of m=23.5 using the 400 megapixel MegaCam camera [CITE]. Observations took place over five seasons, each lasting approximately 5 months. In total 202 nights have been allocated to the survey.

\subsection{Cadence and Observations}
SNLS carried out the observations in four, \textit{griz} photometric bands, similar to the filters used in SDSS (Figure [CITE]). Each field was observed with an average cadence of 5 days and simultaneously observed all filters. The cadence was selected with SN-Ia in mind as the main focus of the survey and would allow for a sufficient coverage of a SN-Ia light curve to perform cosmological fitting. In the late stages of the season where the observations could not be carried out for the same amount of time due to the sun contains some filters have been dropped from the observing schedule. This is because in the late stages the survey is no longer interested in detecting new SN in the blue filters but following up older objects that have been detected before and due to their evolutions are brighter in the redder bands. 

\subsection{Data Reduction}
SNLS data has gone though many separate reductions processed depending on the team that was using the data and the stage of operations. 

\paragraph{Real-time photometry}
During the live operations of the survey the pre-processed data has been independently analyses by the French and Canadian teams using separate quick-reduction pipelines [CITE]. Both teams produced lists of viable SN candidates that have been distributed to the spectroscopic follow-up facilities within a few hours from the observation taking place. 

\paragraph{Forced photometry}
During the survey operation the data has been reduced using a more precise custom force photometry pipeline, PTFPhot, developed by the Canadian team [CITE ROB] which formed the base for the work on the rate of SN-Ia [CITE PERRETT] as well as our work on the rate of SLSNe [CITE CHAPTER 4]. Forced photometry works on the principle of selecting high quality, low Point Spread Function (PSF), reference images which are then downgraded in quality to match that of the science images containing the transients. The reference images must not contain any SN light hence they usually either predate the explosion epoch of the SN use images takes several years after the explosion epoch. Once the science and reference images are matched in image quality they are warped using SWARP [CITE] so that they photometric solutions also match. After this, the reference image is subtracted from the science image. The difference here between the quick reduction photometry and force photometry is that instead of measuring the centroids of the point sources in one image and extracting its photometry the centroids are measured in the entire series of images and then an aperture matching that of the PSF of the image is used to extract the flux count. This largely decreases the uncertainties associated with the flux measurement, especially for faint sources close to or below the detection limit.

\paragraph{Scene Modeling Photometry}
Further to this, the French team has developed a Scene Modeling Photometry (SMP) pipeline to achieve an even higher quality of SN light curves [CITE ASTIER]. This was used in the cosmology studies and, as opposed to forced photometry, does not change the quality of any images. Instead the the galaxy and SN flux are modeled at a resolution higher than the pixel scale before the PSF smoothing is applied to it. These model images are then subtracted from the science frames allowing for similar transient flux measurement to Force Photometry.  
 
\subsection{Spectroscopic Follow-Up}
The success of SNLS cannot be attributed only to the design of the principle component of the survey but also to the effort behind the spectroscopic follow-up of the transient candidates. Between the Very Large Telescope (VLT), Keck, Gemini North and South and Magellan XXX hours have been allocated for spectroscopic follow-up. This in total exceeds the time allocated to SNLS itself. 


\section{Dark Energy Survey}
The Dark Energy Survey (DES) is the largest cosmology survey operated to date. It is composed of two elements; a wide galaxy survey and a SN component. It's main aim is to perform the most precise, to date, measurement of the cosmological parameters. 

\subsection{Survey Setup}
DES uses a purpose build XXX mega-pixel DECam camera mounted on the 4m Blanco telescope at the Cerro-Tollolo observatory in Chile. One of the greatest advantages of DECam over its predecessors is its high sensitivity at red wavelengths allowing for high signal to noise observations in \textit{z} and \textit{y} bands. DES was awarded 500 observing nights over the period of 5 years between 2013 and 2018. Each observing seasons starts in late August and ends at the end of January giving on average a 5 months long observing period.   

\paragraph{Wide Survey}
DES wide survey has been designed to observe 5000 deg$^2$ of the southern sky to the depth of m$\sim$25 in the \textit{ugrizy} filters. The aim of this is to create the largest catalog of galaxies and their associated redshifts to date at an unprecedented resolution and depth with the first public data release contained a catalog of 400 million galaxies from the initial 3 years of the survey. These data can be used to study cosmology via three separate experiments; the Baryon Acoustic Oscillations (BYO), Galaxy Clustering and Weak Lensing [CITE ALL]. 

\paragraph{Supernova Survey}
The SN survey as part of DES is a dedicated search optimized to study and discovered the highest number of SN-Ia to date. To maximize the number of discovered SN-Ia 10 fields are observed in the \textit{griz} photometric bands. Eight of these fields are 'shallow' and observe to the depth of m=23.5 while two fields have been chosen to be 'deep' and observe to m=25. The number of deep and shallow fields was chosen as an optimal value that allowed for the largest number of SN-Ia to be discovered at the same time searching for high redshift objects at z>1 [CITE].

The cadence of the DES 

As the the Weak Lensing experiment of the wide survey requires very good seeing images in order to resolve the shapes of galaxies, it is that component that receives a priority over the best quality night. In case where the observations are 

\subsection{Photometric Redshifts}
With the vast quantities of galaxies detected by DES it is not currently possible, even with the largest integrated field spectrograph, to obtain spectroscopic redshifts for even a small fraction of these objects. A photometric approach must be used to achieve the overall cosmological goals of the survey. While most DES observations are carried out using the \textit{griz} filters \textit{u} and \textit{y} are used, albeit at a lower depth, to aid with the galaxy SED template fitting used to estimate the redshifts. This method is not particularly accurate at low redshifts, however it becomes sufficiently precise at z>0.3 [CITE]. This is due to the steep oxygen[??] drop at 4000\AA [??] shifting into the redder bands amongst other galaxy features. 


\section{SUDSS}
The Search Using DECam for Superluminous Supernovae (SUDSS) (PI: Sullivan) was designed as a dedicated SLSN survey working in conjunction with DES. Its aim was to extend the DES observing season from 6 months to $\sim$ 9 months by observing the fields at high airmass until the observations become impossible due the sun constrains. While, do to technical issued \sref{sec:SUDSSIssues}, it has never been used as a tool for searching for new SLSN candidates it has greatly enhanced the light curves of some SLSNe discovered by DES. 

\subsection{Observations and Data Reduction}
SUDSS observed XX fields between the end of 1st DES season and start of the 4th season (Periods Pxx to Pxx). The observations were carried out using DECam with an identical setup to DES allowing for the same pre-processing steps to be applied to the data. The data was not directly fed into the DES database nor did it use the force photometry pipeline used by DES for technical reasons. Instead the data was reduced using a custome PTFPhot pipeline [CITE ROB]. For the purpose of unbiased comparison, corresponding DES observations of the SUDSS SLSNe were also reduced using the PTFPhot pipeline.

\subsection{Cadance and Exposures}
\label{sec:SUDSSCadance}
SUDSS was designed with the aim of discovering new SLSN candidates at its very heart. It was initially believed [CITE] that it will be capable of identifying $\sim$ 100 SLSN candidates during its operation. As SLSN are a class of slowly evolving supernova, the DES observing cadance was not retained for SUDSS and instead it was chosen to be two weeks. Due to worse that anticipated weather conditions many of these nights have been lost to cloudy nights resuting in a final cadance closer to 1 month. On avarage three additional observations were made in each field outside of the DES observing season.

A futher difference between the DES and SUDSS observing strategies comes in the exposire times used for the observations. As SLSN are much brighter but rarer events than SN-Ia (which are the main focus of the DES SN survey) the exposure times did not have to match that of the DES deep fields to still search for SLSNe at redshifts greater than z>1, however, they did exceed the relatively short exposure times of DES shallow SN field. In practice the limitig magnitude of SUDSS was similar to that of the DES shallow fields once the unfavourable atmospheric conditions and high air mass have been taken into account.

\subsection{Findings}
\label{sec:SUDSSIssues}
As a result of the poor cadance and lower than expected image quality it was not possible to carry out a dedicated search for SLSN in the SUDSS data alone. Instead, the SUDSS data have been repurposed as an auxilary data source for objects already discovered by DES. This has been extremely sucessful and resulted in a great improvemnt to the light curves of some of the SLSNe discoved by DES. In cases of DES15C3hav and DES15X2hm we were able to determine the explosion date and hence contrain the rise time of these SNe using the SUDSS data points.

The most significant observations came in form of DES14X3taz [CITE MAT] where the auxilary SUDSS data provided the observations of the pivotal rise and peak of this unique "bumpy" SLSN. Without the SUDSS observations a large proportion of the analysis carried out in [CITE] would have not been possible.

\section{DES Spectroscopic Follow-up Facilities}

\section{Literature sample of SLSNe}
Throughout this thesis a sample of SLSN, published in years between 2010 and 2016, has been used to define what a SLSN is. We define SLSN as any hydrogen poor SLSN and do not distinguish between slowly and rapidly evolving objects [CITE INSERRA AND GAL-YAM]. All objects come from a variety of surveys which includes xx PTF, 1 iPTF, 2 SNLS, 3 DES and xx PS1 SNe. The broad mix of surveys means that the objects are not of the same quality, are not in the same photometric systems and do not use the same band passes.

\subsection{Quality cuts}
We devised a set of quality cuts to ensure that the objects used in our analysis meet a minimum standard. As most of our analysis, discussed in the forthcoming chapters, depends on fitting black bodies to the light curve we require that the light curve must contain a minimum of three distinct photometric bands, i.e Even thought they would be modeled independently we would count SDSS and PS1 \textit{r}-band filters as one for the purpose of the quality cut. Furthermore we make a cut on the number of epochs observed per band. In order to use models which depend on the rise and decline time of the supernova we require that there are a minimum of two photometric points before the peak and two points after the peak. As the SN evolve rapidly in temperature and hence colour, the bluer bands peak several days before redder bands. We define the peak as the maximum in the band closest to rest frame U-band. \tref{tab:PubishedSLSNe} shows the list of SLSN used throughout this work that have passed our data quality cuts.

\subsection{Converting and converging photometric systems}
It is standard practice in the SN literature to publish the photometry for studied objects in form of a table of magnitudes per band on given observing nights. Magnitude scale is not, however, the most useful when fitting models to the data. As it is logarithmic in nature, the photometric uncertainties, originally measured in the linear count space and often Gaussian in nature become asymmetric and therefore more difficult to deal with in model fitting. It is a lot easier to convert the magnitudes, M, to flux, $\nu$ using a zero point, $zp$ for a given filter:
\begin{equation}
\label{eq:MagToFlux}
\nu = 10^{-0.4 \times M~+~zp}
\end{equation}
similarly we convert the uncertainties in magnitude to flux using \eqref{eq:MagErrorToFlux}.
\begin{equation}
\label{eq:MagErrorToFlux}
\Delta \nu = 0.921034~\times~10^{-0.4 \times M~+~zp}~\times~\Delta M
\end{equation}
It is customary in literature to quote the magnitudes for the SDSS filters (\textit{ugrizy}) in the AB photometric system and the Johnson filters (\textit{UBVRIJHK}) in the Vega system. If not otherwise stated in the papers these are the systems used for the zero points in conversion between magnitude and flux. If not stated otherwise it is also assumed that the photometry has been corrected for Milky Way extinction which is a standard procedure in most modern surveys [CITE PTF, DES, SNLS etc].

\begin{table}
\begin{center}
  \caption{The training sample of SLSNe-I.}
\label{tab:PubishedSLSNe}
\begin{tabular}{|l|r|l|l|}
\hline
  \multicolumn{1}{|c|}{SN Name} &
  \multicolumn{1}{c|}{Redshift} &
  \multicolumn{1}{c|}{Survey} &
  \multicolumn{1}{c|}{Reference} \\
\hline
  PTF12dam & 0.108 & Palomar Transient Factory & \citep{2013Natur.502..346N}\\
  SN2011ke & 0.143 & Catalina Real-Time Transient Survey & \citet{2013ApJ...770..128I}\\
  &&\& Palomar Transient Factor & \\
  SN2010gx & 0.230 & Palomar Transient Factory & \cite{2010ApJ...724L..16P}\\
  SN2013dg & 0.265 & Catalina Real-Time Transient Survey & \cite{2014MNRAS.444.2096N} \\
  PS1-11ap & 0.524 & Pan-STARRS & \cite{2014MNRAS.437..656M}\\
  DES14X3taz & 0.608 & Dark Energy Survey & \cite{2016ApJ...818L...8S} \\
  PS1-10bzj & 0.650 & Pan-STARRS & \cite{2013ApJ...771...97L}\\
  DES13S2cmm & 0.663 & Dark Energy Survey & \cite{2015MNRAS.449.1215P} \\
  iPTF13ajg & 0.741 & intermediate Palomar Transient Factory &\cite{2014ApJ...797...24V}\\
  SNLS-07D2bv & 1.500 & SNLS &\cite{2013ApJ...779...98H}\\
  SNLS-06D4eu & 1.588 & SNLS &\cite{2013ApJ...779...98H}\\
\hline\end{tabular}
\end{center}
\end{table}
