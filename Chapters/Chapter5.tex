\chapter{Classifying SLSN using Machine Learning}
\label{Chapter5}
\lhead{Chapter 5. \emph{ML Classification}}

Throughtout this thesis, I worked with an underlying theme of performing SN classifications, focusing particularly on the selection of SLSN. In \cref{Chapter3}, I establised a definition of SLSNe in terms of the parameter space of the spin-down of a Magnetar model, later used to photometrically classify one SLSNe in the SNLS archival data. In this chapter, I start by describing these techniques as applied to the DES data set during the real-time search for SN in seasons two and three. This included both the manual scanning of SNe and combining it with the Magnetar model fitting. While successful in identifying several, later confirmed, candidate SLSNe, a number of misclassifications highlighted a need for a more robust approach.

In recent years the whole field of astronomy entered a new data analysis renaissance, utilising the Big Data tools and Machine Learning techniques to extract more from archival data and prepare for the arrival of new surveys such as Gaia, LSST, SKA that are expected to produce stagering amounts of data. These are not only difficult to handle for astronomers used to working on much smaller data samples but will infact require absolute state-of-the-art facilies and tools to handle and analyse the data streams. Following this revolution, I endevored to apply some of the latest techniques in ML to the problem of SN classification.

To date, there has been a number of SN studies aiming at classifying SN with the help of ML. Here, I am focusing only on the studies classifying the light curves of SNe and not point source classification pipelines such the once used in DES \citep{Goldstein2015} and Suburu \citep{Morii2016}. Amongst a number of similar works \citep{Karpenka2012,Moller2016,Charnock2016} the most thorough and indepth study of ML classification of SNe is presented in \citet{Lochner2016}. In their work, a number of models are used to extract a range of light curve features. They also provide a comparison of a number of Supervised ML algorithms and discuss their merits in terms of SN classification. While thorough in their analysis, their approach is not ready for deployment in a real survey. The training sampled used in their analysis is the SPCC dataset containing a sanitised sample of SNe only. In a survey such as DES we would first have to separate SNe from other types of transients which, when considering ML techniques, can be performed at the same stage as the classification of SNe.

The use of the SPCC dataset, containg only 2000 SNe weighted to match their observed rates, by \citet{Lochner2016} along with all previous ML classification studies of SNe is one of their greatest drawbacks. In this thesis, I aim to provide the first SN classification study utilising a large (tens of thousands objects of each class), artificially generated sample of objects. With the size of the training sample I tackle the common issue with overfitting for the less common subclasses of SNe. Furthermore, by placing the SNe uniformally throughtout the DES observing season and at a wide range of redshifts, I introduce real survey inperfections including objects whoch suffer from season edge effects and low S/N.

One of the greatest differences between all previous studies and this thesis is the family of ML algorithms used. \citet{Lochner2016} used the SALT2 model, in the process commonly refered to in ML as Feature Extraction or Feature Engineering, to extract a set of parameters that describe the light curve. These are then fed into the machine learning algorithm to produce their classifications. In this thesis, I use Convolutional Neural Networks (CNN), an overwelmingly powerful technique which dominates in the world of commercial ML solutions. CNNs, described in detail in \sref{sec:CNN}, use the data directly as they building their feature sets as part of the learning process. The use of a such a complex ML tool is only possible thanks to the size of the training sample and the data augmentation described in \sref{sec:DataAugmentation}.

In this chapter, I describe the process of creating an artificial training sample of SNe for a majority of subclasses, based on the tools developed in \cref{Chapter4}, as well as AGNs and noise spikes which form the majority of transients detected by DES. I then describe the steps taken to apply the survey noise model to the otherwise smooth simulated data before interpolating and ugmenting it with the help of GPR, described in \sref{sec:GP}. Finally, I discuss the use of the CNN framework to provide a photometric classification for all transients detected by DES in the first four years of its operations.

\section{Search for SLSN in DES using a Non-ML}
Before using machine learning to find SLSN I have used the magnetar model in the same way as in \cref{Chapter2}. This did produce some results and resulted in the classification of a few SLSNe in the data, however, we then started finding objects which really did not fit the model at all and sometimes even had a Chi2 of 3000, as in the case of DES15S2nr. Because of this, we knew that we need to try a different approach.

\subsection{Manual scanning of Transients}
text
\subsection{Magnetar Model Fitting}
text

\section{Training sample} \label{sec:TrainingSample}
In any ML project, the training sample used to build the classification model is its most import element. Regardless of the algorithm used without the correct samples the model will not be able to accurately label new data and often may fall victim to overfitting. An ideal training set would be large compared to the number of distint classes, containing unambiguesly labelled objects and be indistinguishable from the unlabelled test sample that we wish to classify. In reality, this is difficult to achive and often involves manual scanning and classification of the training sample by the user (e.g the manual scanning of point source detection in DES \citep[][and similar studies]{Goldstein2015}).

In the case of SN light curves building the training sample is very difficult. The data comes from a very wide range of sources as the observations are taken using different telescopes, instruments and filters. The raw numbers of classified SNe are also insufficient with only several thousand objects classified to day [CITE?]. While commonly used, the SPCC dataset does not satisfy the requirement for a CNN training sample as the sample is too small.

In this thesis I, therefore, produce an artificial sample of objects that fit all of our requirements from first principles. For each class of object I determine the parameter space for their respective models that can be used to create a large quantity of perfect light curves in the DES photometric bands with an arbitraty cadence. I then apply the DES cadence and noise model to them, creating simulated survey events that closely resembles the real objects.

\subsection{DES Noise Model} \label{sec:NoiseModel}
The noise model applied to the data in this chapter is based heavily on the routines implemented in the SNANA package \citep{Kessler2009}. SNANA is a very powerful package designed as a tool for a realistic light curve simulations. It was used to simulate the SPCC data set \citep{Kessler2010} as well as generating a sample of SN\,Ia used to determine observational biases in the DES SN cosmology study [WHO DO I CITE; in prep.]. However, due to the design of the package, it is very difficult to implement further models into the code.

In DES, SNANA forms the backbone of the SN analysis and is used to extract the image quality logs from the science and reference frames; including the zero points, PSF and sky backgrounds amongst others. Internally these logs are refered to as the \textsc{SIMLIB} files. I follow the same procedure as implemented in SNANA to determine the uncertainty associated with an observation, given its mjd, observing field, filter and the CCD number. The final flux is then drawn randomly from a Normal distribution centered at the simulated flux, with a variance equal to the estimated uncertainty. To demonstrate the effectiveness of this approach, \fref{fig:IaNoiseComp} shows the comparison between the \textit{r}-band light curve of an example SN\,Ia observed by DES and a simulated light curve denerated based on a SALT2 model (performed using the SNCosmo package \citep{Barbary2014}) fit to the original object. The S/N ratio of the observed and simulated light curves fall close to unity, demonstrating their agreement.

\begin{figure}
  % \includegraphics[width=\textwidth]{/path/to/figure}
  \caption{\textit{Top}: \textit{r}-band light curve of an example SN\,Ia observed by DES and a simulated object created to replicate the original data point. \textit{Bottom}: the ratio of the S/N for the observed and simulated data light curves shown to be in agreement.}
  \label{fig:IaNoiseComp}
\end{figure}

\subsection{SN\,Ia}
Amonst the various SN classes simulated in as part of this thesis, SN\,Ia are unquestionably the most well-studied and understood class of objects. Thanks to over two decades of use as cosmological probes, there exists a number of packages able to model and simulate these objects with a high accuracy. Furthermore, the paramater spaces of SN\,Ia as well as peculiar outliries to the class (SN\,Ia-91bg and SN\,Ia-91T) are well understood, gives us a great starting point for our simulations.

While it would be possible to perform our own simulations, starting with any implementation of the SALT2 model (e.g SNANA, SNCosmo or otherwise), and pass these through the DES noise model (\sref{sec:NoiseModel}), I would essentially be replicating the sample of fake SN\,Ia injected into the science images during the live data reduction stages. In DES, these objects are used to estimate the image quality and generate the \textsc{SIMLIB} files making them equivalent to light curve that would be generated through SNANA.

The light curves of SN\,Ia that are injected into the images are generated using the extended SALT2 model as used in \citet{Betoule2014}. The upper redshift range was set as z=1.4 ensuring that the sample is not limited by the simualted redshift. We expect that DES is able to detect SN\,Ia up to redshift z$\sim$1.3. The fake SNe are injected such as to match the rate evolution measured by \citet{Perrett2012}. 50,000 objects have passed the detection criteria and therefore form part of our training sample of SN\,Ia. \fref{fig:IaDist} shows the redshift distribution for the sample.

\begin{figure}
  \includegraphics[width=\textwidth]{Figures/Chapter5/SNIa_z_dist.png}
  \caption{The redshift distribution of SN\,Ia that form part of our training sample. }
  \label{fig:IaDist}
\end{figure}

\subsection{CCSN}
The rate of CCSNe in the local universe is intrinsingly higher than the rate of SN\,Ia [CITE THE BOOK] in the local universe, with an approximate flaction of 70\% CCSNe to 30\% SN\,Ia. However, as CCSN are a fainter with an average brightness only a tenth of the SN\,Ia luminocity. As a result, survey such as DES we can only detect them up to the z$\sim$0.6 as opposed to z$\sim$1.3 for SN\,Ia, giving a much lower observed rate. In our study it was necessary for our sample not to replicate this observed rate as the training of a CNN required that all classes used in the classification are relatively equally represented.

While there is a great diversity amongst CCSNe in terms of both their morphology and peak luminocity, our simulation package, \textsc{CoCo} (\sref{sec:CoCo}), was designed to simulate their whole population. As the overarching aim of this thesis is the classification of SLSNe, as opposed to providing an accurate photometric classification for all transients detected by DES, an accurate treatment of the CCSNe could be considered excesive. The Nickel decay, powering mechanisms behind SN\,Ia and SN\,Ib/c, results in light curves which are often difficult to separate, giving rise to the original inspiration behind developing \textsc{CoCo}. A possible drawback of this could be overfitting of SN\,Ia and CCSNe if these two classes were found to be too alike. Contrarily, I would not wish to indruduce such bias into our training sample as the low redshift behaviour of the SNe may not necessarily be reflected for objects observed at higher redshifts when all observational factors are considered. The behaviour of the objects may also differ between SN\,Ib/c and SN\,II, hence in this chapter I build separate samples for these two classes of SNe both matching the SN\,Ia sample in size. In \sref{sec:CNN} I then consider performing SN classification by mixing these objects at various abundancies.

\subsubsection{SN\,Ib/c}
In \cref{Chapter4}, I presented a method for creating templates as well as simulating SN\,Ib/c. I have used \textsc{CoCo}, the package developed for this purpose to simualate over 100,000 objects based on the spectroscopic templates shown in \tref{tab:IbcTemplates}. Due to the low number statistics of the sample, the spectral subtypes of the templates do not match their measured relative abundance. Following Firth et al. (in prep), I use the \citet{Li2011} measurement of the local rates of CCSNe to correct their abundancies. A question was raised as to whether the subclasses should be represented equally thoughout the sample or match their observed rates. I made the decision to follow their observed abundancies as we are not interested in assigning subclasses to these SNe. Furthermore, some subclasses are very ill-represented in our sample and would therefore be replicated more than other objects likely leading to model overfitting.

\begin{table}
  \caption{}
  \label{tab:IbcTemplates}
  \centering
  \begin{tabular}{l|r|r}
    SN Name  & redshift & count \\
    \hline
    SN1993J  & 0.79 & 1263 \\
    SN1994I  & 0.70 &  313 \\
    SN1996cb & 0.78 &  945 \\
    SN1998bw & 0.80 & 6151 \\
    SN2002ap & 0.80 & 1271 \\
    SN2005bf & 0.80 & 3660 \\
    SN2005hg & 0.80 &  887 \\
    SN2006aj & 0.80 & 5002 \\
    SN2007Y  & 0.57 &  146 \\
    SN2007gr & 0.67 &  345 \\
    SN2007uy & 0.64 &  263 \\
    SN2008D  & 0.28 &   38 \\
    SN2008ax & 0.43 &  208 \\
    SN2008bo & 0.46 &  108 \\
    SN2009iz & 0.80 &  623 \\
    SN2009jf & 0.80 & 1844 \\
    SN2010al & 0.80 & 5008 \\
    SN2011bm & 0.80 & 7554 \\
    SN2011dh & 0.73 &  964 \\
    SN2011ei & 0.39 &  105 \\
    SN2012ap & 0.54 &  191 \\
    \hline
  \end{tabular}
\end{table}

I generate the light curves in the redshift range of 0$<$z$<$0.8, drown using a volume weighted, non-uniform random distribution corresponding to the star formation rate of the universe \citep{Hopkins2006}. To introduce a level of diversity into the sample, I apply host galaxy reddening to the templates (Milky Way extinction is not necessary as DES observed is very low extinction fields). I use the \citet{Cardelli1989} law with R$_\mathrm{v}$=3.7 and the A(B-V) values drawn randomly from the modulus of a Normal distribution centered at zero with a variance, $\mu$=0.2. Furthermore, I have used a similar (but unmodulated) distribution to apply an offset to the peak magnitude of the SNe. This is, again, aimed at introducing a scatter into the training sample as the low number of available templates, despite being placed at different redshifts and explosion dates, would produce a sample of repeating objects that yet again could lead to overfitting.

From the simulations we measure the detectibility of each template as the function of redshift (see \tref{IbcTemplates}) as well as the detection frequency of the whole population as a function of redshift shown in \fref{fig:IbcDist}. As expected SN\,Ib/c are detectable only up to z$\sim$0.8 which is significantly lower than that of SN\,Ia (\fref{IaDist}).

\begin{figure}
  \includegraphics[width=\textwidth]{Figures/Chapter5/SNIbc_z_dist.png}
  \caption{}
  \label{fig:IbcDist}
\end{figure}

\subsubsection{Hydrogen-Rich SN}
The method used for building the training sample of SN\,II followed that of SN\,Ib/c very closely. I used the template sample, generated using the \textsc{CoCo} package, as present in \tref{tab:SNIITemplates}. The only significant difference in the process of generating the sample was the simulated redshift range which was increased to z$\sim$0.9 as we have found during our testing that using z<0.8 would result in a number of objects being detected at the upper redshift limit.

\begin{table}
  \caption{}
  \label{tab:SNIITemplates}
  \centering
  \begin{tabular}{l|r|r}
    SN Name  & redshift & count \\
    \hline
    SN1999el & 0.40 &  4052 \\
    SN2000cb & 0.36 &  4127 \\
    SN2000eo & 0.78 & 20895 \\
    SN2002gd & 0.26 &  1924 \\
    SN2006V  & 0.45 &  7178 \\
    SN2007pk & 0.75 & 18822 \\
    SN2009E  & 0.30 &  3538 \\
    SN2010al & 0.90 & 30882 \\
    SN2011hs & 0.26 &   792 \\
    SN2012ec & 0.36 &  5176 \\
    SN2013ej & 0.36 &  4114 \\
    \hline
  \end{tabular}
\end{table}

From \tref{tab:SNIITemplates} we see that there is a bimodality in the detectibility o the training sample whereas most of the objects are detectable only at low redshhift with three templates being detected at a significantly greater redshift. As a result these objects have a much higher frequency of detection in my simulations, scewing the detection distribution (\fref{fig:IIDist}). This, again, may cause serious overfitting issues. However, as I generate an overabundance of points, in \sref{sec:CNN} I investigate different combinations of the training samples including one where I select it uniformally based on the original template the simulation was based on.

\begin{figure}
  \includegraphics[width=\textwidth]{Figures/Chapter5/SNII_z_dist.png}
  \caption{}
  \label{fig:IIDist}
\end{figure}

\subsection{SLSN}
The simulations of SLSNe have been one of the most challenging parts of this thesis. The low number of known examples of the class, combined with the uncertainty surrounding their definition and the engine powering their staggering luminocity makes the modelling of these objects a challenging task. In the case of their simulations, for the purpose of a classification study such as the one presented in this thesis, we must not only replicate all previosly observed objects but also make a reasonable, but not overly constraining, prediction for what SLSN may appear as and represent as a class. Until very recently, this would have not been possible as our understanding of the models of SLSNe, and their parameters was not sufficient. However, several works including \cref{Chapter3,Chapter4} of this thesis as well as \citet{Inserra2013,Nicoll2013,Nicoll2017} and Angus et al. (in prep) showed that the spin-down of a Magnetar model is able to describe all SLSNe with an acceptable level of accuracy.

The greatest step towards simulating SLSNe was achived in \citet{Inserra2018a} where we have established their new definition that is versitile and robust enough to produce a sample that matches all observed objects but at the same time is limited in the span such that it does not overlap with other classes of SNe. This definition is based on Four Observables Parameter Space (4OPS), defined in narrow (800\AA and 1000\AA wide respectively) box filteres centered at 4000\AA and 5200\AA, by; the peak in the 4000\AA band light curves, colour between the band at peak and +30 days post maximum, and the drop in magnitude between the peak and +30 days in the 4000\AA band. \citet{Inserra2018a} finds that SLSNe form linear correlations in this paramater space, with a narrow scatter shown in \fref{fig:4OPS}. I use this property to determine the magnetar model parameters which correspond to the definition of SLSNe.

\begin{figure}
  % \includegraphics[width=\textwidth]{/path/to/figure}
  \caption{}
  \label{fig:4OPS}
\end{figure}

I simulate a large number of SLSNe using the magnetar model and compare them to the 4OPS definition. The magnetar model parameters are drawn randomly from a uniform top-hat distribution, bound at 10<$\tau_M$<150, 0.1<$B_{14}$<20.0, 0.01<$P_{ms}$<10.0. These limits were informed by the definition of SLSNe used in the rate calculation described in \cref{Chapter3}, but was expanded to ensure completeness but, at the same time, limit the number of computationally expensive simulations that needed to be performed. I have also introduces a modification to the 4OPS definition of SLSNe, similar to that in Angus et al. (in prep), where as the parameter spaces are expanded by one magnitude of scatter while retaining the original slope of their relationship. This was aimed at including fainter objects such as those found in the DES spectroscopically confirmed sample (\sref{sec:DES_SLSN}), that were not available at the time the relationship was constructed. The modified limits are presented in \tref{tab:4OPS} and can be seen in \fref{fig:4OPS} along with the regions where objects drawn from the magnetar model.

\begin{table}
  \caption{}
  \label{tab:4OPS}

\end{table}

\begin{figure}
  % \includegraphics{/path/to/figure}
  \caption{}
  \label{fig:4OPSMag}
\end{figure}

In \fref{fig:4OPSMag}, I show the distribution of magnetar model paramaters that result in objects that match the 4OPS definition of SLSNe. Perhapse the most interesting result is the luminocity function that results from drawing objects uniformly from the magnetar model parameter space. With no external inputs, the functions matches that observed by \citet{DeCia2017} in the PTF sample of SLSNe. I use the sets of magnetar model parameters that match the SLSN definition to generate the training sample of SLSNe for the use in \sref{sec:CNN}. In total $\sim$100,000 SLSN were generated at a range redshifts with 0<z<3, drawn from a volume weighted distribution following the start formation rate of the universe \citep{Beacon2004} similarly to CCSNe (\fref{fig:SLSNDist}).

\begin{figure}
  \includegraphics[width=\textwidth]{Figures/Chapter5/SLSN_z_dist.png}
  \caption{}
  \label{fig:SLSNDist}
\end{figure}

\subsection{AGN}
Active Galactic Nuclei (AGN) are the largest contaminant in the DES sample. This is only partially due to their physical morphology but in greater measure the design of the survey. AGNs are most commonly associated with long, quasi-periodic, variable light curves that are often easy to identify based on their historical variability. As no long-term observations, matching its depth, are available for the DES SN fields making each first AGN detection its discovery. Prior to the first season of DES, a set of science verifiction images were taken which were used as templates in the image supraction pipeline used to detect new transients. AGNs that have undergone rebrightening in the first season were detected as potential supernova candidates. If the templates remained unchanged for the duration fo the survey we would see a decrese in the contamination each season. Furthermore, we would be able to remove most of these transients restrospectivaly by selecting objects with detections in multiple seasons. The survey did, however, change the templates each year in the first three seasons of DES to use the best images taken in the previous season. In the final two seasons the data from year two was used as templates. Additionally, the data for the first season was later reanalysed using season two as a template. A caveat of DES that caused us particular issues is that negative `detections' are not considered as detections within the transient selection algorithms. As a result, each DES season contains a large number of objects which are selected as new transients despite showing stong visual signs of prior, albeit `negative', variability.

In the cases of SLSNe, this is particularly troubling as their slow evolution can be sometimes confused with an AGN with a quasi-period of approximately one year, if only a single DES seasons are considered. It is, therefore, imparitive that AGNs are correctly represented in the ML training sample used in \sref{sec:CNN}. For this purpose, I use the existing simulations of AGNs, in the DES observing bands, presented in \citet{Honig2016}. While these simulations were originally aimed at evaluating the possibility of using AGNs as cosmological probes applying a technique called Reverberation Mapping, they were perfectly suited for this project. The simulated light curves did not contain any observed noise, were densely sampled (one day cadance) and had a span exceeding four years. In total 100,000 simulated objects were available for this study including those placed at redshifts outside the detectible range for DES. I have therefore placed each object at random start dates and fields, before applying the survey noise using the method described in \sref{sec:NoiseModel}, resulting in $\sim$60,000 detected AGNs.

\subsection{Noise}
Visual inspection of the DES light curve data shows that a limited number of objects, originally recognised as a real transient according to the DES transient selection criteria (\cref{Chapter2}), do not appear to be physical in origin. There appear to be two main origins for these objects: bad image subtractions and spurious noise detections. Despite a soffisticalted, ML powered, transient selection pipeline \citep{Goldstein2015}, some objects (often elongated and with negative subractions; \fref{fig:BadSubtractions}) can pass the ML cuts, albeit with a low score. In some cases, likely dependant on the observing conditions, this may occur in several epoch separated by less than 30 days, giving the object a `real transient' flag.

Another channel that can result in the misclassifications of candidates is the detection of slow-moving, near earth object. Such objects are most commonly detected at the same position only in two or three consecutive bands. In subsequent epochs, they are, in normal circumstances, no longer detected at the same position resulting in the object being rejected. However, in some rare cases, bad subtractions or random noise spikes exceeding the 5 sigma detection limit, within 30 days cut can result in a `real transient' flag.

\begin{figure}
  % \includegraphics{/path/to/figure}
  \caption{}
  \label{fig:BadSubtractions}
\end{figure}

To model these objects, I use a very simple approach of inserting a number of sharp, $\delta$-function like spikes in the data that correlated between the filters and separated by less than 30 days to account for the misidentifications. The spikes are selected between 19$<$mag$<$22 in order to test the different behavious of the GP interpolations used in the next step of this analysis (\sref{sec:DataAugmentation}). The absolute value of the peak does not play an important role in the classification process as the data is normalised before entering the CNN. Similarly to other classes of transients, $\sim$100,000 objects have been simulated across all fields.

\subsection{Missing classes}
While in this thesis, I have created one of the most thorough training samples of SNe for the purpose of a ML classification study, it still cannot be said to be complete. There are a number of classes of known transient objects which we were unable to accounted for in this work. Some objects such as kilonovae, associated with gravitational waves as their optical counterparts, evolve too rapidly to be using the cedence of DES. Omitting these (undetectable) events does not bring any difference to the final result of our classification. However, one class of objects which could have an effect on our final classification are the newly discovered class of rapidly evolving SNe \cite{Drout2014,Kepler2018}. Recent work by \citet{Pursiainen2018} showed that these objects are relatively common within DES with 72 detections in the first four seasons of its operations. While there are now early, tentitive, signs \citep{Pursiainen2018} that these objects are powered by a SN shock interacing within a thick an extended wind \citep{Piro2015}, similar to the model used in modelling of the `bumps' found in SLSNe \sref{sec:MagExtensions}. It is this particular connections that would be interesting to explore, however, the modelling of these objects is still in its infancy. The model parameter spaces defining them, their SED models and other simialar aspects developed for SLSNe over the last several years have not yet been studied for the fast evolving SNe.

The omission of rapidly evolving SNe from our sample will result in these objects being mislabelled in our classifications. It would however be unlikely that any of the objects that we have not included in the training sample would get mislabelled as a SLSNe due to their significanly faster evolution. I test this assertion in \sref{CNN} using the ground-truth sample of objects identified by \citet{Pursiainen2018}.

\section{Data Augmentation} \label{sec:DataAugmentation}
Before the training sample of SNe created in \sref{sec:TrainingSample} can be used to build a CNN classification model, it has to first undergo a number of augmentation steps. The data passed through the CNN must be uniform in terms of the number of points as well  their separation, regardless of the field and season the data was taken from. To achieve this, I first select a length of observations, then I choose a new cadance that will overlap most closely with that observed by DES before applying the flux correction required to normalise the effect of using a changing subtraction template in different season. Finally I use GPR in order to interpolate and augment the data such as to meet the requirements of CNN.

\subsection{Choosing the observing block}
I define the observing block, for the used in our classification study, as the span of time (measured in days) that is the longest period over which an DES season observes any of its field. Due to the observing conditions and sheduling the DES observing seasons vary between the seasons and fields, with a difference can between the longest and shortest observing window measuring 40 days. This is not only caused by the sheduling or the observability of the fields but also due to the observing conditions as in the early parts of the season a large number of observations is lost to the cloudy conditions. This results in either large gaps in the data for all or several filters.

Selecting the observing block is not as straight foward as to simply use the length of the shortest season. \fref{fig:ObsBlock1,fig:ObsBlock2,fig:ObsBlock3,fig:ObsBlock4} show the cedance of DES in seasons 1-4 in all fields and bands.

This is done in two stages, first I find the span that covers all the filters. From the first point where all the filters are observed to the last point where all the filters are observed. I then remove the points at the start of the season where there is a gap in the data longer than 10 days. This happens in a few filters where an observation has been missed due to the atmospheric conditions. If a similar gap to this exists later on in the light curve the GPR interpolation can handle it well but at the start of the season there is not enough information to anchor the fit.

From these measurements I determine the optimal observing block to be 149 days in duration. The span covered by the block in each season and filter is shown in \fref{fig:ObsBlock1,fig:ObsBlock2,fig:ObsBlock3,fig:ObsBlock4}.

\begin{figure}[h]
  \includegraphics[width=\textwidth]{Figures/Chapter5/ObsBlock_Season1.pdf}
  \caption{}
  \label{fig:ObsBlock1}
\end{figure}

\begin{figure}[h]
  \includegraphics[width=\textwidth]{Figures/Chapter5/ObsBlock_Season2.pdf}
  \caption{}
  \label{fig:ObsBlock2}
\end{figure}

\begin{figure}[h]
\includegraphics[width=\textwidth]{Figures/Chapter5/ObsBlock_Season3.pdf}
  \caption{}
  \label{fig:ObsBlock3}
\end{figure}

\begin{figure}[h]
  \includegraphics[width=\textwidth]{Figures/Chapter5/ObsBlock_Season4.pdf}
  \caption{}
  \label{fig:ObsBlock4}
\end{figure}

\subsection{Choosing the cadence}
Upon deciding on the observing block used in the simulations the next step I applied was was to decide on the cadance of the interpolated observations. As I am using GPRs I could apply essentially any cedence I want and there is no real rule for how much data should be included in the sample. I did however not want to either loose any information or use too many data points which would esentially only replicate the information that we already have.

The observed cadence of DES is not uniform and varies through the season due to the observing conditions. Early in the season, the cadence is shorter as the obseving weather conditions don't allow for the observatins of the wide DES fields leading to a shorter DES cadence as the deep and shallow fields are given more observing time. On the contrary, the cadence stabilised as the season progresses and settles at the designed 7 days. This is seen in \fref{fig:cadence} resulting in a bimodal distribution with an average cadance of 5 days. With a lack of any other factors that could help us decide on the cadence for the interpolated I use this value in \sref{sec:UseGP}.

\begin{figure}
  \includegraphics{Figures/Chapter5/Cadence.pdf}
  \caption{}
  \label{fig:cadence}
\end{figure}

\subsection{Applying Flux correction to Real Data}
Before the GPR can be applied to all the data it first needs to be corrected for the effects of DES switching the image subtraction template between season. As the focus of the DES SN team is predominantly the study of SN\,Ia they have always focused on maximising the quality of single season light curves. As the image quality has improved in the first two seasons of observations the templates have been updated each year. This is not an for the study of short transients where only a single season is of interest, however, for slowly evolving SNe (and AGNs) this causes an issue where the supernova light curve is present in the template result in decreased flux in the subsequent season. This is a particular problem for SLSN, where their evolution can often be slow enough to be detectable in multiple seasons in DES (Angus et al.; in prep). This may potentially be a strong factor in their classification and hence cannot be ignored.

While the most optimal approach to this would be to perform the image subraction and source detection with a single template this was both computationally prohibitive due to the scale and complexity of the raw DES data. As an alternative solution, I used the DES analysis logs to determine which observations have been used in the creation of the template images. In these frames, I measure the median flux of the object and use this value to correct the offset. As a simple test for this approach, I use a light curve of a SN\,Ia that explosed early in the second season of DES. In the uncorrected DES light curves this results in a flat but negative light curve in the third season. \fref{fig:FluxOffset} shows the original and corrected light curve, consistant with zero flux in the third season.

\begin{figure}
  % \includegraphics{/path/to/figure}
  \caption{}
  \label{fig:FluxOffset}
\end{figure}

\subsection{Applying GPs} \label{sec:UseGP}
The size and extend of the training sample used in this thesis is one of the two advancements made towards classifying SNe in DES using the ML approach. An equally important step, crucial for the use with CNNs, was the use of GPR (\sref{sec:GP}) as a tool for light curve interpolation and augmentation. CNN performs pattern recognision using a set of convolutional kernels with a fixed size, therefore requiring the data to be evenly sampled. This has tremendous benefits as it does not require any feature extraction steps and uses every data point in the classification process. While the observed data cannot be used directly in with this technique due to its non-uniform sampling, using a GP interpolated light curves removes less information about the data than a parametric model. Simulataniously, it does not introduces any correlations between the distinct bands, both in terms of the flux and the onset of the SN.

I apply the method described in \sref{sec:GP} to interpolate the light curves using the Mate\'rn 3/2 covarience functions.

\section{Classifications} \label{sec:CNN}
text
\subsection{Convolutional Neural Networks}
text
\subsection{Feeding the data}
text
\subsection{SNe vs AGN vs Noise}
text
\subsection{Classifying SNe}
text
\subsection{SLSNe in DES}
