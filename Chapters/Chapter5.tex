% Chapter 5

\chapter{Classifying SLSN using Machine Learning} % Write in your own chapter title
\label{Chapter5}
\lhead{Chapter 5. \emph{Machine Learning Classification}} % Write in your own chapter title to set the page header

\section{Non-ML approach}
Before using machine learning to find SLSN I have used the magnetar model in the same way as in \cref{Chapter2}. This did produce some results and resulted in the classification of a few SLSNe in the data, however, we then started finding objects which really did not fit the model at all and sometimes even had a Chi2 of 3000 like in the case of DES15S2nr. Because of this we knew that we need to try a different approach. 

\section{Machine Learning}
Machine Learning is the modern approach to problems of classification and broadly data analysis. It has been used throughout SN science for a while now mainly in surveys as a tool for distinguishing between real detections and artifacts in astronomical images. This has been very successful [CITE PTF, DES etc]. 

\subsection{SN Classification using ML}
In recent years attempts have been made to use ML techniques for photometrically classifying SN from their light curves. The current state of the art work was done by \citet{Lochner2016PHOTOMETRICLEARNING} which attempted classification of SN-Ia vs CCSN as well as discriminating between different CCSN subtypes. The project did however suffer from a small training sample used in the analysis, with only several hundred SN-Ia and as few as 3 SN-Ib! The training sample used was the Supernova Photometric Classification Challenge (SPCC) \citep{Kessler2010} which was used to develop techniques of photometric classification of SN-Ia for the purpose of the DES cosmology project. The results were not optimal and left a lot of room for improvement inspiring a number of groups to try to improve on this work including this thesis.

Another challenge that has never been tackled by any publication in the field is the classification of SN types not simply amongst a perfect sample of confirmed SN (Like in the case of using the SPCC

\section{Data Augmentation}
