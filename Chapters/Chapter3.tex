\chapter{Rate of SLSNe at z$\sim$1}
\label{Chapter3}
\lhead{Chapter 3. \emph{Rate of SLSNe}}

In this chapter I present the measurement of a volumetric rate of SLSNe at z$\sim$1 measured using the archival SNLS data. First I propose a photometric definition of a SLSN based on the modeling of light curves using the spin-down of a magnetar model followed by a search for missed objects within the SNLS dataset. I briefly introduce the new photometric sample before describing the Monte-Carlo simulation of the survey, aimed at measuring the rate and its uncertainties. Finally, I compare the value measured to those found in the literature and discuss the significance of the result in terms of the physical origin of the objects as well as their detectability by DES and other, future, surveys. The work presented in this chapter has been published in \citet{Prajs2016}.

\section{Defining a SLSN}
Our first task in the process of measuring the rate of SLSNe was to develop a method for the photometric selection of optical transients that were to enter our SLSN sample. The earliest definitions of SLSNe relied on their peak luminosity and often defined them as SN with an absolute $u$-band magnitude of $M_{u}<-21$ \citep{2012Sci...337..927G}. It was our view that this definition is not adequate for our search for two main reasons. The first was that, already at that time, there existed several examples in the literature of events that are spectroscopically similar to SLSNe, but that do not pass this arbitrary threshold. Some of these events included DES13S2cmm \citep{2015MNRAS.449.1215P} and PTF11rks \citep{2013ApJ...770..128I} which both had M$_u$ > -21 but are also associated with a very slow evolution which is synonymous with SLSNe. The second reasons was the discovery of new classes of fast and luminous transients \citep{2016ApJ...819...35A} with luminosities similar to SLSNe, but with a faster light curve evolution and different spectroscopic types. Therefore, instead of an arbitrary luminosity cut, we use a photometric classification approach based on modeling SLSNe using the spin-down of a magnetar model as a simple analytical model that provides a good fit to a sample of confirmed SLSNe.

This approach did, however, have its drawbacks. Use of an analytical model implied that the rate was calculated only for objects which are similar to the sample of confirmed SLSNe, as presented in \sref{sec:litSample}. We were aware that this would likely not capture all objects of this class at the time this work was undertaken but without a better definition of a SLSN more significant progress could not be achieved. In \cref{Chapter5} I present a more significantly more complex approach to this problem, as applied to DES, based on an artificial training sample of SLSNe and Machine Learning classifiers.

\subsection{Modeling SLSNe}
In order to model SLSN we required two main ingredients: an underlying model for the time-dependent bolometric luminosity of a SLSN, and an SED that can convert this bolometric luminosity into time-evolving spectra. This together forms a spectral series from which we can synthesis the photometry in any desired filter and at any desired redshift. This can then be used for comparison to observed data. For the purpose of this chapter the exact form or the physical meaning behind the light curve model parameters are irrelevant as, in theory, any model that satisfactorily replicates the observables could be used for this purpose. I therefore provide only a brief overview of the model here while the remaining details are described in \sref{sec:SLAP}.

The spin-down of a magnetar model assumes that a collapse of a giant star with an initial mass of M $>$ 40M$_{\odot}$ produces a SNIc like explosion whilst giving birth to a magnetar; a highly magnetized neutron star with a rotation period on the time scale of milliseconds \citep{Kasen2009,Woosley2010,Inserra2013}. The simple toy model assumes a spherically symmetric ejecta and a constant opacity. The free parameters in the model are the Magnetic field, $B$, spin period, $P$ and the diffusion timescale $\tau_M$ which is directly proportional to the ejecta mass, M$_{ej}$.      

\section{Searching for SLSN in SNLS}
\subsection{Magnetar model fitting}
\subsection{Candidates}
\subsection{SNLS04D3bs}

\section{SNLS Monte-Carlo}

\section{Rate of SLSN}

\section{Overview}
